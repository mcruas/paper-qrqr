\section{Concluding Remarks}

In this work, we propose a nonparametric methodology based on QR to build a CDF for a wind time series which we name QRAL model. QRAL, is by construction, a linear model with regularization both on the quantiles and on the covariates, with the aim of producing proper conditional scenarios for several quantiles at once. These scenarios are input for various applications in power systems and are essential in measuring risk in energy trading, planning the expansion of the energy systems and in dispatch. The nonparametric framework for estimating the CDF is specially useful when data is not distributed according to a known probability distribution. Our results show that the scenarios generated through QRAL have desirable qualities as measured by criteria such as proximity with unconditional quantiles (GFS) and attaining both goodness of fit and parsimony (SIC), outperforming other known benchmarks, as QR model(Koenker's original model) and SARIMA.  The results obtained encourages us to consider further improvements in our methodlogy, as for example, different strategies for regularization between quantiles, and the investigation of the method in higher frequency data, namely, daily and hourly data.

% \todoi{Ajeitar as referências na mão para a Versão Final}
