\section{Concluding Remarks}

In this work, we propose a nonparametric methodology based on QR to build the CDF of wind time series.  Two different approaches for QR are developed: one is a linear model with regularization both on the quantile and on the covariates to produce future scenarios of RG, while the other is nonparametric on the quantiles and flexible enough to capture a nonlinear relation with the covariates. These scenarios are input for various applications in power systems and are essential in measuring risk in energy trading, planning the expansion of the energy systems and the dispatch problem. The nonparametric framework for estimating the CDF is specially useful when the data is not distributed according to a known distribution. The scenarios generated outperformed other known benchmarks, both in the QR literature (the Koenker model) as in the classic time series framework (SARIMA). The results obtained in this work brings incentives continue researching methods for nongaussian time series, as they are present in many real world applications. 