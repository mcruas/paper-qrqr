\section{conditional distribution based on Quantile Regression for Time Series}

In the previous section, we presented a linear model for estimating a single $\alpha$-quantile by using QR with adaLASSO as a regularization strategy to select the best model. However, to build a CDF from an array of quantiles, we propose to jointly estimate them, in order to explore the connection across different quantiles. 

Let the finite discretization of the interval $[0,1]$ be composed of a sequence of probabilities $0 < \alpha_1 < \alpha_2 < \dots < \alpha_{|J|} < 1$ and denote as $A$ the set $A = \{ \alpha_j \mid j \in J \}$, where $J$ is an index set for the probabilities $\alpha$. 
The $\alpha$-quantiles are, from this point forward, indexed by $j$, to account for the different models that are simultaneously estimated. A property that must be respected is the monotonicity of the quantile function $Q$, such that $q_{\alpha_1} \leq q_{\alpha_2} \dots \leq q_{\alpha_{|J|}}$.
The sequence of quantiles defines a continuous quantile function after interpolation, and finally a CDF after inverting the estimated quantile function.


% todo rever essa parte , daqui até o fim da seção

In order to produce a distribution function, the output of the problem must respect certain properties, such as being monotonically increasing. 
If a sequence of quantiles do not respect such property, the issue is known as crossing quantiles.
Besides monotonicity, one can expect that the value of similar quantiles be produced by similar models. 
If the coefficient of a given $p$ covariate changes too abruptly with respect to a change on the probability $\alpha$, there is a high probability that the estimation did not produce good results, capturing noise on the data.  
To account for all these issues, all quantiles must be estimated at once. 

In order to induce smoothness of coefficients across quantiles, we introduce a second derivative filter, given by the discrete approximation shown below:
\begin{equation}
D_{\alpha_j}^{2} \beta_{pj} := \frac{\left(\frac{\beta_{p,j+1}-\beta_{pj}}{\alpha_{j+1}-\alpha_{j}}\right)-\left(\frac{\beta_{p,j}-\beta_{p,j-1}}{\alpha_{j}-\alpha_{j-1}}\right)}{\alpha_{j+1}-\alpha_{j-1}}. 
\end{equation}
The works by \cite{zou_regularized_2008, jiang_interquantile_2014} also use multiple quantile regressions at once and make use of interquantile similarities to produce regularization on the quantiles. In \cite{zou_regularized_2008}, the authors use the norm $\| \beta \|_{1\infty}=\sum_{p=1}^{|P|} \max\{ |\beta_j^{(k)} |\}$ as penalization. Such penalization is imposed on the maximum value among all quantiles for a given covariate. This idea is extended by \cite{jiang_interquantile_2014}, that uses a fused AdaLASSO mixing the $\ell_1$ penalization with the absolute interquantile difference.

The statistical model \textbf{Quantile Regularized Adaptive LASSO (QRAL)} is defined by the vector of coefficients $\beta_{0}$ and the matrix of size $|P| \times |J|$ of regressor coefficients $\beta_{pj}$, which are the solution of the minimization problem given below:
% \begin{IEEEeqnarray}{lr}  % para duas colunas
% 	\underset{\beta_{0j},\beta_j}{\text{min}} \sum_{j \in J} \left( \sum_{t\in T}\rho_{\alpha_j}(y_{t}-(\beta_{0j} + \beta_j^T x_t)) \right.  \span \nonumber \\
% 	\span \left. + \lambda\  \sum_{p \in P} w_{pj}^\delta \mid  \beta_{pj} \mid \right) + \gamma \sum_{p \in P} \sum_{j \in J'} |D^2_{pj}|,\label{eq:adalasso_model}
% \end{IEEEeqnarray}
\begin{IEEEeqnarray}{lr} % para duas colunas
  \underset{\beta_{0j},\beta_j}{\text{min}} \sum_{j \in J} \left( \sum_{t\in T}\rho_{\alpha_j}(y_{t}-(\beta_{0j} + \beta_j^T x_t)) \right. \span \nonumber \\  
  \span \left. + \lambda\    \sum_{p \in P} w_{pj}^\delta \mid  \beta_{pj} \mid \right) + \gamma \sum_{p \in P} \sum_{j \in J'} |D^2_{\alpha_j}\beta_{pj}|, \label{eq:adalasso_model_mat1}\\
  \text{subject to:} \span \nonumber \\
	\beta_{0j} + \beta_{j}^T x_{t} \leq \beta_{0,j+1} + \beta_{j+1}^T x_{t},& \forall t \in T, \forall j \in J_{(-1)}, \label{eq:adalasso_model_mat2} 
\end{IEEEeqnarray}
where the weights $w_{pj} = 1/\tilde{\beta}_{pj}$ and $\tilde \beta_{pj}$ are the coefficients from the first-step LASSO estimation. The parameter $\delta$ is an exponential parameter usually set to 1.
The sum of absolute values that compose the second derivative filter $\sum_{j \in J'}\sum_{p \in P}|D_{\alpha_j}^{2}\beta_{pj}|$ is added on the objective function multiplied by a tuning parameter $\gamma$, where the set $J'=\{2,\dots,|J|-1 \}$.
With this approach, one can keep track of the aforementioned  crossing quantiles issue  
as well as using an interquartile structure as a strategy to reduce noise on estimation. % todo  explicar melhor o porquê 
