\section{Introduction} \label{sec:introduction}

% = 1. Talk about renewable energy and its variability - motivacao da importancia do estudo da probability forecasting OK

%Load forecasting for renewable energy is a topic of growing interest in the academic community, fueled by the new challenges in modeling uncertainty in such a setting, which is mostly explained by the intermittent nature of renewable sources.    %It is a much cleaner way of producing energy than by using other sources such as coal and gas, and with less hazard potential than nuclear power plants. 

%The installed capacity of renewable energy plants has been increasing in a fast pace and projections point out that wind power alone will account for 18\% of global power by 2050 \cite{IntEnerAgency}.
%In spite of its virtues, several new challenges are inherent when forecasting with such power source, due to its unpredictability. To overcome this lack of certainty, one has to work with many different possibilities of outcome.

%Many applications in Power Systems use renewable scenarios as input.
%For all the aforementioned applications, the knowledge of the time series conditional distribution can provide all that needed information.

Renewable generation (RG) forecasting is a topic of growing interest in the power-system academic community due to the massive number of applications that benefit from accurate forecasts. The intermittent nature of the RG and the complexity of the power-system applications require specific and challenging developments, within which usual point forecasting methods are being more and more deprecated in favor to more sophisticated probabilistic forecasting approaches. Such a probabilistic forecasts are in general used to improve decisions with risk-analysis-based information relying on the description of extreme quantiles \cite{pinson_probabilistic_2009}. Hence, new statistical models capable of handling such difficulties are an emerging field in power systems literature. See, for example, \cite{bessa2012time, gallego2016line,moller_time-adaptive_2008,nielsen2006,bremnes_probabilistic_2004,wan_direct_2017}. The main objective in such a literature is to propose new models capable of generating scenarios to approximate the joint probability distribution of RG time series for various applications in power system based on stochastic optimization models \cite{BirgeBook2011}. For instance, applications in energy trading \cite{Fanzeres2015}, unit commitment \cite{papa2013,papa2015}, grid expansion planning and investment decisions \cite{Aderson2017,munoz2015,moreno2019}) are relevant and timely examples where scenarios play a crucial role. 


In \cite{zhang_review_2014}, the commonly used methodologies regarding wind power probabilistic forecasting models are reviewed, and split into parametric and nonparametric classes. The main characteristics of \emph{parametric models} are (i) assuming a distribution shape and (ii) low computational costs. The ARIMA-GARCH model, for example, fits the RG series by assuming a Gaussian distribution \emph{a priori}. On the other hand, \emph{nonparametric models} have the following characteristics: (i) do not require a specified distribution, (ii) require more data to produce a good approximation and (iii) have a higher computational cost. Popular methods are quantile regression (QR), kernel density estimation,  artificial intelligence, or a combination of them.
%%%%%% 3. Falar da não-gaussianidade do WP e apresentar novas referências
% Unir com parágrafo anterior??
% Não gaussianidade dos dados de Fator de capacidade eólico 3 paragrafo HHH
Most time series methods rely on the assumption of Gaussian errors. However, RG time series such as wind and solar are reported as non-Gaussian \cite{bessa2012time,jeon2012using,taylor2015forecasting,Wan2017}. To circumvent this problem, the usage of nonparametric methods - which do not rely on a distribution assumption - appear as promising alternatives. 

To simulate scenarios, one needs the entire conditional distribution. The importance of characterizing the whole distribution becomes even more relevant in applications with asymmetric costs such as those found in power systems applications, where, e.g., load shed costs are in general much higher than the cost of spilling renewable energy. Hence, having a good estimate of the conditional distribution function (CDF) is essential for meeting accurate estimates of the risk involved in operational and planning decisions \cite{Fanzeres2015}. 


The seminal work \cite{koenker1978regression} defines Quantile Regression (QR) as the solution of an optimization problem where the sum of the ``check" function $\rho$ (defined formally in the next session), a piecewise linear convex function, is minimized. Conditional quantiles are the result of the above problem, and this approach is employed in many works \cite{chao_quantile_2012,li_quantile_2007,bosch_convergent_nodate,gallego2016line,moller_time-adaptive_2008,nielsen2006,bremnes_probabilistic_2004,wan_direct_2017}. Applications are enormous, ranging from risk measuring at financial funds (Value at Risk) to a central measure robust to outliers.




% However, by simply joining an array of quantiles
%QR is a tool for constructing a methodology for non-gaussian time series, because of its facility to implement on commercial solvers and to extend the original model.

% However, when estimating a distribution function, as each quantile is estimated independently, the monotonicity of the distribution function may be violated.
% This issue - also known as crossing quantiles - can be adressed by constraining the sequence of quantiles to be in an increasing order. Other possibility is making a transformation afterwards, as shown in \cite{chernozhukov_quantile_2010}.


%
%, as defined in \cite{koenker_quantile_2006}.

%%%%% 4. Falar de regressao quantilica em geral. Onde é utilizada e etc.


% In \cite{koenker_quantile_2006}, the application of QR is extended to time series, when the covariates are lagged values of $y_t$.  
%In our work, beyond autoregressive terms, it is also considered other exogenous variables as covariates. 



%%%%% 5. aplicações de QR em wind power, colocando os papers mais proximos.
% colocar tb regressao quantilica com regularizacao
In \cite{gallego2016line,moller_time-adaptive_2008,nielsen2006,bremnes_probabilistic_2004,wan_direct_2017}, QR is employed to model the conditional distribution of wind power time series.
An updating quantile regression model is presented in \cite{moller_time-adaptive_2008}. The authors present a modified version of the simplex algorithm to incorporate new observations without restarting the optimization procedure.
%Using existing wind power forecasting to extend these forecasting to build a model of quantiles is the strategy adopted by \cite{nielsen2006}.
In \cite{nielsen2006}, the authors build a quantile model from an already existent independent wind power forecasts. The approach by \cite{gallego2016line} is to use QR with a nonparametric model. The authors add a penalty term based on the reproducing kernel Hilbert space, which allows a nonlinear relationship between the explanatory variables and the output. That work also develops an on-line learning technique, where the model is easily updated after each new observation arrives.
In \cite{wan_direct_2017}, wind power probabilistic forecasts are made by using QR with a special type of neural network (NN) with one hidden layer, called an extreme learning machine. In this setup, each quantile is a different linear combination of the features of the hidden layer. The authors of \cite{cai_regression_2002} use a weighted Nadaraya-Watson kernel estimator to estimate a conditional density function from a given time series.


The objective of this work is to propose an adaptive nonparametric CDF-based model driven by a jointly-estimated array of quantile regressions in past data. The goal is to approximately derive the full conditional-distribution of the renewable data to generate synthetic time-series scenarios through Monte Carlo simulation. To do that, the quantile space is discretized within a user-defined granularity and linear interpolation is used in between to derive the full CDF. Instead of having an independent model for each quantile, quantiles are jointly estimated by a unique mathematical optimization problem, whose aim is to estimate the coefficients of the set of auto-regressive quantile functions that would later form a CDF after interpolation. The proposed optimization problem used to estimate the model ensures that the estimated CDF preserves monotonicity (through non-crossing quantile constraints), and that the regression coefficients are smooth across quantiles. This link among quantile values and coefficients creates the idea of a single distribution model, which in the optimization problem is imposed by regularization terms. This is one of the salient properties of our model.

More objectively, in our work, the QR is estimated using a novel approach called Quantile Regularized Adaptive LASSO (QRAL). QRAL relies on a quantile autoregressive (QAR) framework in the same spirit of \cite{koenker1978regression,koenker_quantile_2006,koenker2005quantile}. To capture the variables that would improve the model fit from a set of candidate variables, we propose the use of a LASSO cost function to select the lags to be included in the model. Regularization is a topic already explored in previous QR works. The work by \cite{belloni_l1-penalized_2009} defines the properties and convergence rates of QR when adding a penalty proportional to the $\ell_1$-norm to conduct a variable selection, using the same idea as the LASSO \cite{tibshirani1996regression}. The adaLASSO equivalent to QR has had its properties investigated by \cite{ciuperca_adaptive_2016}. In this variant, the penalty for each variable has a different weight, and this modification ensures that the oracle property is being respected. In \cite{zou_regularized_2008,jiang_interquantile_2014}, the adaLASSO is employed in QR with multiple quantile regressions, using the inter-quantile dependence to improve quantile coefficients estimation. 

Based on the parsimonity principle, we expect that the same coefficient for the same auto-regressive lag should not drastically change within small variations in the quantile probability. This issue was addressed in \cite{jiang_interquantile_2014}. In this paper, the absolute value of the discrete first derivative of coefficients across the quantiles were penalized within a multi-quantile regression applied to cross-sectional data. However, the aforementioned penalization is known to produce stepwise-shaped filtered signals \cite{kim2009ell_1}, contradicting the idea of a single continuous inter-quantile model for the CDF. Hence, an innovative feature of our work is the consideration of a $\ell_1$-filter, proposed by \cite{kim2009ell_1}. Based on this framework, we use a regularization term that produces a continuous (piecewise linear) shape for the parameters across the quantiles within a multi-quantile auto-regression model applied to a RG time series. To do that, instead of penalizing the first derivative, we penalize the absolute value of the second derivative of each auto-regressive coefficient across the quantile probabilities. %Hence, based on this second regularization scheme, we achieve one of the contributions of our work, namely, a method to estimate a CDF based on jointly-regularized quantile auto-regressive models.

Interestingly, in general, statistical estimation procedures focus on minimizing point forecasting errors. However, the link between minimal forecasting error estimation methods and high-quality scenarios, properly characterizing the distributions simulated some steps ahead, is not direct. Additionally, most of power system applications rely on a probabilistic forecast \cite{pinson_probabilistic_2009,papa2015,Aderson2017,Fanzeres2015,gallego2016line}. Hence, to determine the best regularization parameters leading to an accurate probabilistic forecast up to $K$ steps ahead, we use a rolling-horizon out-of-sample evaluation procedure whereby the distance between the empirical distribution and the predictive CDF is minimized. Finally, an out-of-sample evaluation test is performed to assess the quality of the proposed model against benchmarks. 
%Although one could argue that if the model is correct and sufficient data is available, the maximum likelihood method should always induce to all possible desirable properties as the true model is recovered from the estimation process. However, in this work, we recognize that the underlying true models driving the RG time-series are highly-complex and only an approximation to them are possible. Hence, when estimating based on one metric, some other desirable properties may deteriorate. This is consistent with the modern data-science approach, where the out-of-sample performance may be adjusted depending on a given application.
 
%%%%%% 7. Objetivos do work e contribuição
%The objective of this work is, therefore, to propose a new methodology to address the simulation of future RG scenarios. This methodology is nonparametric and builds the CDF from an array of jointly estimated quantiles, using a penalty that helps estimate a more appropriate CDF. More precisely, 
The contributions of this work are twofold:
\begin{itemize}
	\item An adaptive nonparametric CDF-based time-series model to simulate RG scenarios through a Monte Carlo procedure. The proposed model is conceived based on a jointly-estimated array of quantile regressions with two regularization terms. While the first considers the adaLASSO method to select the explanatory variables (auto-regressive terms and/or exogenous variables), the second imposes a coherence link between all the many simultaneous quantile models to form a predictive CDF.
	
	\item A linear optimization model to jointly-estimate the proposed nonparametric CDF-based time-series model that ensures the aforementioned coherence property for the CDF. This is done by the consideration of a $\ell_1$-filter, proposed by \cite{kim2009ell_1}, where the absolute value of the second derivative, across the quantile probabilities, of the auto-regressive coefficients is penalized.
	
\end{itemize}

%We propose a new combination of methods to predict the $k$-step ahead conditional distribution. By using MILP, we achieve a solution which is optimal for the given objective. In order to improve the quality of predictions and interpretability, we incorporate a joint regularization by specifying the existence of groups among the probabilities $\alpha$. We could not find any other work in the literature that interpreted different quantiles as models depending on one another. 
%The objective of this work is to propose and test different techniques of predicting the conditional distribution based on QR. 


% OBJETIVOS:
%Um modelo para séries tmeporais autoregressivo e nao parametrico e baseado na função quantilica. No caso autoregressivo, uma metodologia de estimação com seleção parcimoniosa otima global é proposta e possibilita o controle dos números de grupos de regressores diferentes dentro do modelo para diferentes quantis. Para o modelo não paramétrico
%Modelo data driven, empirico

The remainder of this work is organized as follows. In Section II, we present the quantile regression framework and the quantile regularized adaptive LASSO (QRAL). In Section III we discuss how to jointly estimate the QRAL in order to form a CDF. Section IV shows how to estimate, evaluate and simulate the model. In Section V, we present two computational experiments: i) a controlled test with a known autoregressive model to illustrate the methodology functioning and ii) a case study based on real data from a Brazilian wind farm. Section VI concludes this study.
% \todoi{Inserir sumário das próximas seções}
%%%%%% 8. Organização dos próximos capítulos OK
