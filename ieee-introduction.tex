\section{Introduction} \label{sec:introduction}

% = 1. Talk about renewable energy and its variability - motivacao da importancia do estudo da probability forecasting OK

Load forecasting for renewable energy is a topic with growing interest by the academic community, fuelled by the new challenges in modeling uncertainty in such setting, mostly explained by the intermittent nature of renewable sources.    %It is a much cleaner way of producing energy than by using other sources such as coal and gas, and with less hazard potential than nuclear power plants. 

%The installed capacity of renewable energy plants has been increasing in a fast pace and projections point out that wind power alone will account for 18\% of global power by 2050 \cite{IntEnerAgency}.
%In spite of its virtues, several new challenges are inherent when forecasting with such power source, due to its unpredictability. To overcome this lack of certainty, one has to work with many different possibilities of outcome.

%Many applications in Power Systems use renewable scenarios as input.
%For all the aforementioned applications, the knowledge of the time series conditional distribution can provide all that needed information.
New statistical models capable of handling such difficulties are an emerging field in power systems literature. See, for example, \cite{bessa2012time, gallego2016line,moller_time-adaptive_2008,nielsen2006,bremnes_probabilistic_2004,wan_direct_2017}. 
The main objective in such literature is to propose new models capable of generating scenarios of renewable generation (RG) time series which are demanded in applications such as energy trading, unit commitment,  grid expansion planning, and investment decisions (see \cite{moreiraStreet,jabr2013robust,zhaoguan,Aderson2017}). 
In stochastic optimization, problems such as unit commitment, economic dispatch, transmission expansion planning all use scenarios as input. 
Such scenarios are used to characterize the probability distribution within the optimization under uncertainty framework.
When working with robust optimization, bounds for probable ranges of coefficients are needed.

%%%%%% 1.b. Continuar 
%It is important to have good forecasts of either high and low quantiles to to measure the probability of extreme scenarios. 
%the complex behavior of wind is very difficult to model and predict.  \todo{Melhorar esta parte? "é importante prever bem quantis altos e baixos p analise de risco - fica prejudicada pela dificuldade de previsao destes quantis"}
%Having better prediction models can help the planner to make better and less risky decisions, increasing the attractiveness of renewable energy to the energy system. 
%In this work we will investigate how to model dynamics of renewable energy time series in both short and long terms.
 
%%%%%%%% 2. Falar de wind power nos primordios. ARIMA e essas coisas
% Henrique faz Critics about point forecasts and gaussian models (ARIMA-GARCH). Compare GAS and nonparametric models.

%% Eu faço a introdução ao probabilistic forecasting
Conventional statistical models are often focused on estimating the conditional mean of a given random variable. % This is not very useful when dealing with renewable energy, as the variability and the notion of risk is extremely important for planning. - ver a distribuicao como um todo - reescrever
%One of the first works in wind power prediction, \cite{brown_time_1984} treated the nonlinearity of wind power by applying a transformation on the prediction of wind speed, which is modeled by an autoregressive process. The data is standardized to account for the normal variation during the day.
%\cite{moeanaddin_numerical_1990} estimated the $k$-step-ahead conditional density function using the Chapman-Kolmorov relation. The method is applied on a non-linear autoregressive time series.
By reducing the outcome to a single statistic, important information about the time series random behavior is lost. In order to account for the process inherent variability, it is important to consider probability forecasting.
In \cite{zhang_review_2014}, the commonly used methodologies regarding wind power probabilistic forecasting models is reviewed, splitting them in parametric and nonparametric classes. The main characteristics of \emph{parametric models} are (i) assuming a distribution shape and (ii) low computational costs. The ARIMA-GARCH model, for example, fits the RG series by assuming \emph{a priori} a Gaussian distribution. On the other hand, \emph{nonparametric models} have the following characteristics: (i) do not require a distribution to be specified, (ii) needs more data to produce a good approximation and (iii) have a higher computational cost. Popular methods are quantile regression (QR), kernel density estimation,  artificial intelligence or a mix of them.
%%%%%% 3. Falar da não-gaussianidade do WP e apresentar novas referências
% Unir com parágrafo anterior??
% Não gaussianidade dos dados de Fator de capacidade eólico 3 paragrafo HHH
Most time series methods rely on the assumption of Gaussian errors. However, RG time series such as wind and solar are reported as non-Gaussian \cite{bessa2012time,jeon2012using,taylor2015forecasting,Wan2017}. To circumvent this problem, the usage of nonparametric methods - which does not rely on distribution assumption - appears as a promising alternative. 

In order to simulate scenarios one needs the whole conditional distribution. For example, if two random variables $X_1$ and $X_2$ have the same mean but the distribution of $X_2$ has fatter tails, then simulations from $X_2$ will present more extreme values than simulations from $X_1$. The knowledge of the scale becomes as important as the knowledge of the location when producing future scenarios. The procedure used for simulating future scenarios is by drawing, in each period $\tau$, a value for $\tau+1$ from the estimated conditional distribution function (CDF). Hence, having a good estimate of the CDF is essential to meet the goal in this work. 

The seminal work \cite{koenker1978regression} defines Quantile Regression (QR) as the solution of an optimization problem where the sum of the ``check" function $\rho$ (defined formally in the next session) is minimized. Conditional quantiles are the result of the above problem, and this approach is employed in many works \cite{chao_quantile_2012,li_quantile_2007,bosch_convergent_nodate,gallego2016line,moller_time-adaptive_2008,nielsen2006,bremnes_probabilistic_2004,wan_direct_2017}. Applications are enormous, ranging from risk measuring at financial funds (Value at Risk) to a central measure robust to outliers.




% However, by simply joining an array of quantiles
%QR is a tool for constructing a methodology for non-gaussian time series, because of its facility to implement on commercial solvers and to extend the original model.

% However, when estimating a distribution function, as each quantile is estimated independently, the monotonicity of the distribution function may be violated.
% This issue - also known as crossing quantiles - can be adressed by constraining the sequence of quantiles to be in an increasing order. Other possibility is making a transformation afterwards, as shown in \cite{chernozhukov_quantile_2010}.


%
%, as defined in \cite{koenker_quantile_2006}.

%%%%% 4. Falar de regressao quantilica em geral. Onde é utilizada e etc.


% In \cite{koenker_quantile_2006}, the application of QR is extended to time series, when the covariates are lagged values of $y_t$.  
%In our work, beyond autoregressive terms, it is also considered other exogenous variables as covariates. 



%%%%% 5. aplicações de QR em wind power, colocando os papers mais proximos.
% colocar tb regressao quantilica com regularizacao
In \cite{gallego2016line,moller_time-adaptive_2008,nielsen2006,bremnes_probabilistic_2004,wan_direct_2017}, QR is employed to model the conditional distribution of wind power time series.
An updating quantile regression model is presented by \cite{moller_time-adaptive_2008}. The authors present a modified version of the simplex algorithm to incorporate new observations without restarting the optimization procedure.
%Using existing wind power forecasting to extend these forecasting to build a model of quantiles is the strategy adopted by \cite{nielsen2006}.
In \cite{nielsen2006}, the authors build a quantile model from an already existent independent wind power forecasts.
The approach by \cite{gallego2016line} is to use QR with a nonparametric model. The authors add a penalty term based on the reproducing kernel Hilbert space, which allows a nonlinear relationship between the explanatory variables and the output. That work also develops an on-line learning technique, where the model is easily updated after each new observation arrives.
In \cite{wan_direct_2017}, wind power probabilistic forecasts are made by using QR with a special type of neural network (NN) with one hidden layer, called extreme learning machine. In this setup, each quantile is a different linear combination of the features of the hidden layer.
The authors of \cite{cai_regression_2002} use a weighted Nadaraya-Watson kernel estimator to estimate a conditional density function from a given time series.


We propose a nonparametric methodology to estimate the CDF given autoregressive terms, with the goal of generating future scenarios of RG time series. This methodology is based on interpolating linearly individual quantiles, where each quantile is estimated using the quantile regression framework.
Instead of having an independent model for each quantile, quantiles are jointly estimated by an unique model, whose aim is to estimate quantiles that would later form a CDF after interpolation. This ensures not only that the estimated CDF preserves monotonicity, but also that the regression coefficients are smooth across quantiles. This link among quantile coefficients creates the idea of a single distribution model. As a consequence, out-of-sample predictive accuracy must improve.
In our work, the QR is estimated using a novel approach named Quantile Regularized Adaptive LASSO (QRAL).
QRAL relies on a quantile autoregressive (QAR) framework in the same spirit of \cite{koenker1978regression,koenker_quantile_2006,koenker2005quantile}. In order to capture which variables would improve model fit, from a set of candidate variables, we propose the use of a LASSO cost function to select the lags to be included in the model. Regularization is a topic already explored in previous QR works. The work by \cite{belloni_l1-penalized_2009} defines the properties and convergence rates of QR when adding a penalty proportional to the $\ell_1$-norm to conduct variable selection, using the same idea as the LASSO \cite{tibshirani1996regression}. The adaLASSO equivalent to QR has its properties investigated by \cite{ciuperca_adaptive_2016}. In this variant, the penalty for each variable has a different weight, and this modification ensures that the oracle property is being respected. % todo citar belloni apecendo o nome(?) 
In \cite{zou_regularized_2008,jiang_interquantile_2014}, the adaLASSO is employed in QR with multiple quantile regressions, using the interquantile dependence to improve  quantile coefficients estimation.
As a second innovative feature of our work, we propose the inclusion of a penalization parameter on the absolute value of the second derivative of the quantile function across the quantile probabilities. 
This technique is the same as the $\ell_1$-filter proposed by \cite{boyd2011distributed}. 
Therefore, such term acts as a filter to impose coefficient stability for a given covariate across a set of quantiles. Such strategy is a key aspect for connecting the estimation procedure among quantiles and producing a coherent distribution model.
To the best of the authors knowledge, no other work has developed a methodology where regularization and estimation of the conditional distribution using QR is carried on at the same time with the objective of generating future scenarios for RG time series.

As a regularization strategy we incorporate parameter selection as a way of selecting the best model for simulation. In order to accomplish this goal, we develop a metric where we compare the quantiles of scenarios with historic quantiles of these periods. These scenario quantiles are taken in a time span where they do not depend on the conditional moment of the time series at the end of the observational period. 
 
%%%%%% 7. Objetivos do work e contribuição
The objective of this work is, therefore, to propose a new methodology to address simulation of future RG scenarios. This methodology is nonparametric and builds the CDF from an array of jointly estimated quantiles, using a penalty which helps to estimate a more appropriate CDF. More precisely, the main contributions of the proposed methodology are:
\begin{itemize}
	\item Estimation of conditional quantiles through a linear function of both past observations and current and past exogenous variables, where selection of variables is achieved through regularization via adaLASSO, which can be cast as a linear programming algorithm;
	\item  Jointly estimation of several conditional quantiles, with built in smoothness on the coefficient values across a set of quantiles and avoidance of quantile crossing;
	\item Development of a selection procedure that use scenarios goodness of fit metric to select parameters. This procedure evaluates how scenarios produced by a given model can recover historic quantiles.
	
\end{itemize}

%We propose a new combination of methods to predict the $k$-step ahead conditional distribution. By using MILP, we achieve a solution which is optimal for the given objective. In order to improve the quality of predictions and interpretability, we incorporate a joint regularization by specifying the existence of groups among the probabilities $\alpha$. We could not find any other work in the literature that interpreted different quantiles as models depending on one another. 
%The objective of this work is to propose and test different techniques of predicting the conditional distribution based on QR. 


% OBJETIVOS:
%Um modelo para séries tmeporais autoregressivo e nao parametrico e baseado na função quantilica. No caso autoregressivo, uma metodologia de estimação com seleção parcimoniosa otima global é proposta e possibilita o controle dos números de grupos de regressores diferentes dentro do modelo para diferentes quantis. Para o modelo não paramétrico
%Modelo data driven, empirico

% \todoi{Inserir sumário das próximas seções}
%%%%%% 8. Organização dos próximos capítulos OK
The remainder of this work is organized as follows. In Section II, we present the quantile regression framework and the quantile regularized adaptive LASSO (QRAL). In Section III we discuss how to jointly estimate the QRAL in order to form a CDF. Section IV shows how to estimate the model, as well as computational issues regarding the estimation and evaluation. In Section V, in order to test our methodology, we present few controlled studies with simulated data and a final case study using real data from wind power. Section VI concludes.
