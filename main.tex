%% Versão final IEEE
 \documentclass[journal]{IEEEtran}


% Versão edição no LATEX
% \newcommand{\CLASSINPUTtoptextmargin}{2cm}
% \newcommand{\CLASSINPUTbottomtextmargin}{2cm}
% \documentclass[letter, onecolumn, 12pt]{IEEEtran} %journal


% Versão para impressão com margem maior para escrita
%  \documentclass[article,onecolumn, 11pt]{IEEEtran}
%  \usepackage[a4paper, marginparwidth=0.3in, 11pt]{geometry}


% \documentclass[journal,12pt,onecolumn]{IEEEtran}
% \usepackage[a4paper, marginparwidth=0.5in]{geometry}

% Versão Rascunho com uma coluna
%\documentclass[journal,12pt,onecolumn,draftclsnofoot,]{IEEEtran}
%\usepackage[a4paper, marginparwidth=1in]{geometry}

% TODO 
% Li agora, 

% Tá ficando Leal! 

% Da uma revisado nos parágrafos antes e depois da métrica do MAE, já lá pro final. O parágrafo antes tinha uns errinhos de escrita. Fale da ideia de que o objeto é ajustar os hyperparametros de regularização objetivando um algo que nenhum modelo faz, que é reproduzir quantis históricos. Enfatizando que simulação é diferente de previsão. 

% The chosen regularization hyperparameters are the ones which introduce 


% Tire a figura com a simulação. Deixe só a tabela. Os cenários não estão acompanhando bem. Estão gordos em agosto e setembro. 

% Vamos mandar pra revisão de inglês. 

% Veja com a Ana, como fazer para o dee poder pagar. 

% Abs! 



\usepackage[utf8]{inputenc}
% \usepackage{epstopdf}
%\usepackage[spanish]{babel}
\usepackage[cmex10]{amsmath,nccmath}
%\interdisplaylinepenalty=2500
\usepackage{amsfonts}
\usepackage{amssymb}
\usepackage{graphicx}
\usepackage{verbatim}
\usepackage{array}
%\usepackage{multirow}
% \usepackage{subcaption}
\usepackage{dcolumn}
\usepackage{color}
\usepackage[noadjust]{cite}
\usepackage{url}
\usepackage{balance}
\usepackage[usenames,dvipsnames]{xcolor}
\usepackage{accents}
\usepackage{enumerate} % http://ctan.org/pkg/enumerate
\usepackage{blindtext}
%\usepackage{soul}
\usepackage[normalem]{ulem}
\usepackage{pgf,tikz}
% \usepackage{algpseudocode} 
% \usepackage{subfigure}
\usepackage{booktabs}
\usetikzlibrary{arrows}
%\DeclareGraphicsExtensions{.eps} % eu que comentei
%\AppendGraphicsExtensions{.pdf} % eu que comentei


% % % Adicionados por mim
%\usepackage[colorinlistoftodos]{todonotes}
\usepackage[colorinlistoftodos,prependcaption,textsize=small]{todonotes}
\usepackage{xargs} 

\newcommandx{\todoi}[2][1=]{\todo[linecolor=Plum,backgroundcolor=Plum!25,bordercolor=Plum,#1,inline]{#2}}
%\newcommand{\todop}[1]{\todo[textsize=tiny]{#1}}
%%%%



\DeclareMathOperator*{\Max}{max}
\DeclareMathOperator*{\Min}{min}
\DeclareMathOperator*{\argmin}{arg\,min}
\DeclareMathOperator*{\Maximize}{Maximize}

\begin{document}
\title{An Interquantile Regularized Time Series Model for Wind Power Probabilistic Forecasting}

\author{Marcelo~Ruas,~%,~\IEEEmembership{Student Member,~IEEE,}
	 	Alexandre~Street,~%\IEEEmembership{Member,~IEEE}	
		and Cristiano Fernandes
	
	%\thanks{This work was partially supported by UTE Parna\'iba Gera\c{c}\~ao de Energia S.A. through R\&D project ANEEL PD-7625-0001/2013.}
	%\thanks{Bruno Fanzeres, Arthur Brigatto and Alexandre Street are with the Electrical Engineering Department, Pontifical Catholic University of Rio de Janeiro (PUC-Rio), Rio de Janeiro, RJ, Brazil (e-mail: \mbox{bsantos@ele.puc-rio.br}; \mbox{street@ele.puc-rio.br}).}
	%\thanks{L. A. Barroso are with PSR Consulting, Rio de Janeiro, RJ, Brazil (e-mail: \mbox{luiz@psr-inc.com}).}
}

\maketitle


\begin{abstract}

Producing probabilistic forecasts for renewable generation has become an important topic in recent power system applications. In this work, we focus on generating future scenarios of wind power generation.  Most time series methods used to produce such forecasts rely on the assumption of a conditional Gaussian distribution. However, real life applications on power systems require proper modeling of non-Gaussian time series with a focus on distribution tail dynamics to address risk-analysis.  
We develop, in this work, a nonparametric methodology based on Quantile Regression to estimate the conditional distribution function for wind time series.  
The conditional distribution is formed by connecting an array of quantiles jointly estimated, for which we develop the model Quantile Regularized Adaptive LASSO (QRAL).	In QRAL, we implement two types of regularization. One is based on the Adaptive LASSO for selecting covariates within each Quantile Regression. 
The other emphasizes the connection across different quantiles, reflecting the fact that quantile functions should have a smooth variation,
exploring the similarity of neighbor quantiles.	A case study with realistic wind power data from the Brazilian Northeast compares the developed models with different benchmarks in the literature. The model was able to outperform them in terms of minimizing the Mean Absolute Percentile Error of future scenarios in recovering the historic monthly quantiles.
\end{abstract}

\begin{IEEEkeywords}
	Quantile auto regression, non parametric model, wind power forecasting
\end{IEEEkeywords}


%\listoftodos


% ===== Sec. I - Introduction ===== %

\section{Introduction}

%%%%%%% 1. Talk about renewable energy and its variability - motivacao da importancia do estudo da probability forecasting OK

Renewable energy power is an emergent topic which is demanding attention from the academic community. %It is a much cleaner way of producing energy than by using other sources such as coal and gas, and with less hazard potential than nuclear power plants. 
The installed capacity of renewable energy plants has been increasing in a fast pace and projections point out that wind power alone will account for 18\% of global power by 2050 \cite{IntEnerAgency}.
In spite of its virtues, several new challenges are inherent when dealing with such power source, due to its unpredictability. To overcome this lack of certainty, one has to work with many different possibilities of outcome.

%Many applications in Power Systems use renewable scenarios as input.
%For all the aforementioned applications, the knowledge of the time series conditional distribution can provide all that needed information.
New statistical models capable of handling such difficulties are an emerging field in power systems literature \cite{bessa2012time, gallego2016line,moller_time-adaptive_2008,nielsen2006,bremnes_probabilistic_2004,wan_direct_2017}. 
The main objective in such literature is to propose new models capable of generating scenarios of renewable generation (RG) which are demanded in (i) energy trading, (ii) unit commitment, (iii) grid expansion planning, and (iv) investment decisions (see \cite{moreiraStreet,jabr2013robust,zhaoguan,Aderson2017} and references therein). 
In stochastic optimization, problems such as Unit Commitment, Economic Dispatch, Transmission Expansion Planning all use scenarios as input. 
Such scenarios are used to characterize the probability distribution within the optimization under uncertainty framework.
When working with robust optimization, bounds for probable ranges of coefficients are needed.

%%%%%% 1.b. Continuar 
%It is important to have good forecasts of either high and low quantiles to to measure the probability of extreme scenarios. 
%the complex behavior of wind is very difficult to model and predict.  \todo{Melhorar esta parte? "é importante prever bem quantis altos e baixos p analise de risco - fica prejudicada pela dificuldade de previsao destes quantis"}
%Having better prediction models can help the planner to make better and less risky decisions, increasing the attractiveness of renewable energy to the energy system. 
%In this work we will investigate how to model dynamics of renewable energy time series in both short and long terms.
 
%%%%%%%% 2. Falar de wind power nos primordios. ARIMA e essas coisas
% Henrique faz Critics about point forecasts and gaussian models (ARIMA-GARCH). Compare GAS and nonparametric models.

%% Eu faço a introdução ao probabilistic forecasting
Conventional statistical models are often focused on estimating the conditional mean of a given random variable. % This is not very useful when dealing with renewable energy, as the variability and the notion of risk is extremely important for planning. - ver a distribuicao como um todo - reescrever
%One of the first works in wind power prediction, \cite{brown_time_1984} treated the nonlinearity of wind power by applying a transformation on the prediction of wind speed, which is modeled by an autoregressive process. The data is standardized to account for the normal variation during the day.
%\cite{moeanaddin_numerical_1990} estimated the $k$-step-ahead conditional density function using the Chapman-Kolmorov relation. The method is applied on a non-linear autoregressive time series.
By reducing the outcome to a single statistic, important information about the time series random behavior is lost. In order to account for the process inherent variability, it is important to consider probability forecasting.
In \cite{zhang_review_2014}, the commonly used methodologies regarding probabilistic forecasting models is reviewed, splitting them in parametric and nonparametric classes. The main characteristics of \textbf{parametric models} are (i) assuming a distribution shape and (ii) low computational costs. The ARIMA-GARCH model, for example, fits the RG series by assuming \textit{a priori} a Gaussian distribution . On the other hand, \textbf{nonparametric models} have the following characteristics: (i) do not require a distribution to be specified, (ii) needs more data to produce a good approximation and (iii) have a higher computational cost. Popular methods are Quantile Regression (QR), Kernel Density Estimation,  Artificial Intelligence or a mix of them.
%%%%%% 3. Falar da não-gaussianidade do WP e apresentar novas referências
% Unir com parágrafo anterior??
% Não gaussianidade dos dados de Fator de capacidade eólico 3 paragrafo HHH
Most time series methods rely on the assumption of Gaussian errors. However, RG time series such as wind and solar are reported as non-Gaussian \cite{bessa2012time,jeon2012using,taylor2015forecasting,Wan2017}. To circumvent this problem, the usage of nonparametric methods - which does not rely on assuming any distribution - appears as a new alternative. 

In order to simulate scenarios, not only the conditional mean is needed, but the whole conditional distribution. For example, if two random variables $X_1$ and $X_2$ have the same mean but the distribution of $X_2$ has fatter tails, then simulations from $X_2$ will present more extreme values than simulations from $X_1$. The knowledge of the scale becomes as important as the knowledge of the location when producing future scenarios. The procedure used for simulating future scenarios is by drawing, in each period $\tau$, a value for $\tau+1$ from the estimated conditional distribution function (CDF). Hence, having a good estimate of the CDF is essential to meet the goal in this work. 

A possible construction of the CDF nonparametrically is by using a sequence of conditional quantile values. By estimating many quantiles on a thin grid of probabilities over the interval $[0,1]$, one can have as many points as desired from an estimated CDF. The second step would be transforming this set of points into a continuous function by interpolation. 
These quantile values may be estimated by a technique such as Quantile Regression (QR). 
The seminal work \cite{koenker1978regression} defines QR as it is employed in many works \cite{chao_quantile_2012,li_quantile_2007,bosch_convergent_nodate,gallego2016line,moller_time-adaptive_2008,nielsen2006,bremnes_probabilistic_2004,wan_direct_2017}. The conditional quantile is the solution of an optimization problem where the sum of the check function (defined formally in the next session) is minimized. Instead of using classical regression to estimate the conditional mean, QR determines any quantile from the conditional distribution. Applications are enormous, ranging from risk measuring at financial funds (the Value-at-Risk) to a central measure robust to outliers.




% However, by simply joining an array of quantiles
%QR is a tool for constructing a methodology for non-gaussian time series, because of its facility to implement on commercial solvers and to extend the original model.

% However, when estimating a distribution function, as each quantile is estimated independently, the monotonicity of the distribution function may be violated.
% This issue - also known as crossing quantiles - can be adressed by constraining the sequence of quantiles to be in an increasing order. Other possibility is making a transformation afterwards, as shown in \cite{chernozhukov_quantile_2010}.


%
%, as defined in \cite{koenker_quantile_2006}.

%%%%% 4. Falar de regressao quantilica em geral. Onde é utilizada e etc.


% In \cite{koenker_quantile_2006}, the application of QR is extended to time series, when the covariates are lagged values of $y_t$.  
%In our work, beyond autoregressive terms, it is also considered other exogenous variables as covariates. 



%%%%% 5. aplicações de QR em wind power, colocando os papers mais proximos.
% colocar tb regressao quantilica com regularizacao
In \cite{gallego2016line,moller_time-adaptive_2008,nielsen2006,bremnes_probabilistic_2004,wan_direct_2017}, QR is employed to model the conditional distribution of wind power time series.
An updating quantile regression model is presented by \cite{moller_time-adaptive_2008}. The authors present a modified version of the simplex algorithm to incorporate new observations without restarting the optimization procedure.
%Using existing wind power forecasting to extend these forecasting to build a model of quantiles is the strategy adopted by \cite{nielsen2006}.
In \cite{nielsen2006}, the authors build a quantile model from already existent independent wind power forecasts.
The approach by \cite{gallego2016line} is to use QR with a nonparametric method. The authors add a penalty term based on the reproducing kernel Hilbert space, which allows a nonlinear relationship between the explanatory variables and the output. This work also develops an on-line learning technique, where the model is easily updated after each new observation arrives.
In \cite{wan_direct_2017}, wind power probabilistic forecasts are made by using QR with a special type of neural network (NN) with one hidden layer, called extreme learning machine. In this setup, each quantile is a different linear combination of the features of the hidden layer.
The authors of \cite{cai_regression_2002} use a weighted Nadaraya-Watson to estimate the conditional function in the time series.

Regularization is a topic already explored in previous QR papers.
The work by \cite{belloni_l1-penalized_2009} defines the properties and convergence rates for QR when adding a penalty proportional to the $\ell_1$-norm to perform variable selection, using the same idea as the LASSO \cite{tibshirani1996regression}. The ADALASSO equivalent to QR has its properties investigated by \cite{ciuperca_adaptive_2016}. In this variant, the penalty for each variable has a different weight, and this modification ensures that the oracle property is being respected. 
In \cite{zou_regularized_2008,jiang_interquantile_2014}, the AdaLASSO is employed to QR with multiple quantile regressions at the same time, relating the estimated coefficients of different quantiles.


We propose a nonparametric methodology to estimate the CDF given autoregressive terms, with the goal of generating future scenarios of RG. This methodology is based on interpolating individual quantiles, where each quantile is estimated via Quantile Regression framework.
Instead of having an independent model for each quantile, they are all connected in a single problem. All quantiles are jointly estimated by an unique model, whose aim is to estimate quantiles that would later form a CDF after interpolation. This ensures not only that the estimated CDF preserves monotonicity, but also that it is smooth, increasing the out-of-sample predictive assertivity.
In this work, the QR may be estimated using an approach developped in this work named Quantile Regularized Adaptive LASSO (QRAL).


QRAL relies on a quantile autoregressive (QAR) framework in the same spirit of \cite{koenker1978regression,koenker_quantile_2006,koenker2005quantile}. Notwithstanding, as an innovation from the referenced works, in order to capture which variables would improve model fit, from a set of candidate variables, we propose the use of a LASSO cost function to select the lags to be included in the model.
Regularization is a topic already explored in previous QR works.
The work by \cite{belloni_l1-penalized_2009} defines the properties and convergence rates for QR when adding a penalty proportional to the $\ell_1$-norm to perform variable selection, using the same idea as the LASSO \cite{tibshirani1996regression}. The AdaLASSO equivalent to QR has its properties investigated by \cite{ciuperca_adaptive_2016}. In this variant, the penalty for each variable has a different weight, and this modification ensures that the oracle property is being respected. % todo citar belloni apecendo o nome(?)
In \cite{zou_regularized_2008,jiang_interquantile_2014}, the AdaLASSO is employed to QR with multiple quantile regressions at the same time, using the interquantile dependence to improve  quantile coefficients estimation.
As a second innovative feature of this work, we propose the inclusion of a penalization parameter of the second difference of quantile values. We argue that such strategy avoids extreme quantiles to be shrinked to zero as fast as the central ones. Such term acts as a filter to impose coefficient stability for a given covariate across the set of quantiles.
For the best of the authors knowledge, no other work has developed a methodology where regularization and estimation of the conditional distribution using QR is carried on at the same time with the objective of generating future scenarios for WPG time series.






%The crossing quantile issue is solved by introducing a constraint on the optimization problems that forces the quantile function monotonicity. Furthermore, in the quantile regression literature for wind forecasting, a sequence of quantiles is provided as output. In our work, we propose to estimate the conditional distribution as a whole.


 
%%%%%% 7. Objetivos do work e contribuição
The objective of this work is, then, to propose new methodologies to address simulating future RG scenarios. These methodologies are nonparametric and build the CDF from an array of jointly estimated quantiles, using a penalty which helps to estimate a more appropriate CDF. Two different forms of estimating quantiles are proposed: using linear models and nonparametric quantile regression. The methodology may be seen as a multiple quantile regression problem that specifies a time series model based on the empirical conditional distribution. The main contributions of this work are:
\begin{itemize}
	\item A nonparametric methodology to model, from a set of quantile estimations, the conditional distribution of RG time series to produce future scenarios.
	
	\item The proposition of a procedure to jointly estimate quantiles. It is a linear model that selects the global optimal solution with parsimony both on the selection of covariates as on the quantiles. This methodology is based on the Adaptive LASSO for QR (Linear Programming). 
	
	\item Regularization techniques applied to an ensemble of quantile functions to estimate the conditional distribution, solving the issue of non-crossing quantiles. On regularizing quantiles, we propose a smoothness on the coefficient value across the sequence of quantiles.
	%\item A nonlinear QR
	
\end{itemize}

%We propose a new combination of methods to predict the $k$-step ahead conditional distribution. By using MILP, we achieve a solution which is optimal for the given objective. In order to improve the quality of predictions and interpretability, we incorporate a joint regularization by specifying the existence of groups among the probabilities $\alpha$. We could not find any other work in the literature that interpreted different quantiles as models depending on one another. 
%The objective of this work is to propose and test different techniques of predicting the conditional distribution based on QR. 


% OBJETIVOS:
%Um modelo para séries tmeporais autoregressivo e nao parametrico e baseado na função quantilica. No caso autoregressivo, uma metodologia de estimação com seleção parcimoniosa otima global é proposta e possibilita o controle dos números de grupos de regressores diferentes dentro do modelo para diferentes quantis. Para o modelo não paramétrico
%Modelo data driven, empirico

% \todoi{Inserir sumário das próximas seções}
%%%%%% 8. Organização dos próximos capítulos OK
The remainder of this work is organized as follows. In section II, we present the quantile regression framework and the Quantile Regularized adaptive LASSO. In section III, we discuss the estimation procedures for them. The regularization strategies are also presented on this section. Finally, in section IV, we present a few controlled studies with simulated data and a final case study using real data from wind power is presented in order to test our methodology. Section V will conclude this article.


% ===== Sec. II - Quantile Regression ===== %
%
\section{Quantile Regression based time series model} \label{sec:qr1}

Let the conditional quantile function of $Y$ for a given value $x$ of the $d$-dimensional random variable $X$, $Q_{Y|X}:[0,1] \times \mathbb{R}^d \rightarrow \mathbb{R}$, be defined as %(in short, from now on, $Q_{Y|X}(\cdot, \cdot)$)
	\begin{equation}
	Q_{Y|X}(\alpha,x) = F_{Y|X=x}^{-1}(\alpha) = \inf\{y: F_{Y|X=x}(y) \geq \alpha\}.
	\label{eq:quantile-function}
	\end{equation}
The function $Q$ is the inverse of the distribution function $F$, and represents the smallest value $y$ for which the distribution function is greater than a given probability $\alpha$.

Let $\rho$ be the check function 
	\begin{equation}\label{eq:check-function}
	\rho_{\alpha}(x)=\begin{cases}
	\alpha x & \text{if }x\geq0\\
	(\alpha - 1)x & \text{if }x<0
	\end{cases}.
	\end{equation}
The quantile function for a given finite sample $\{y_t,x_t \}_{t \in T}$, where $T$  is the set of time indexes, and a given probability $\alpha$ is the solution of minimizing the loss function $L_\alpha(\cdot)$:
	\begin{IEEEeqnarray}{C}
	\hat{Q}_{Y|X}(\alpha,\cdot) \,\, \in \,\,  \underset{q\in \mathbb{Q}}{\text{arg min}}\, L_\alpha(q) = \sum_{t\in T}\rho_{\alpha}(y_{t}-q(x_t)).\label{eq:optim-lqr1} 
	\end{IEEEeqnarray}
For inference on QR and finite sample properties, see the Chapter 3 on \cite{koenker2005quantile}.
The $\alpha$-quantile function $q(\cdot)$ belongs to a function space $\mathbb{Q}$. We might have different assumptions for space $\mathbb{Q}$, depending on the type of function we want to find 
for $q$. A few properties, however, must be achieved by our choice of space, such as being continuous and having limited first derivative. In this work, only the case where $\mathbb{Q}$ is a linear function space is considered. 
Thus, $q$ is of the form $$q_\alpha(x_t) = \beta_0 + \beta^T x_t.$$ 

% The linear model presumes that the $\alpha$-quantile function is a linear function of its regressors:
% $$q_{\alpha}(x_t) = \beta_{0} + \beta^T x_t.$$   
When dealing with many candidates of covariates, one has to properly select the relevant ones out of a set of candidates. In practice, it means that some coefficients from the vector $\beta = [ \beta_{1 } \cdots \beta_{p} ]$ might assume value zero.
There are many ways of selecting a subset of variables among the available options.
A classical approach for this problem is the Stepwise algorithm \cite{efroymson1960multiple}, \cite{hocking_selection_1967}, \cite{tibshirani1996regression}, which includes new variables iteratively. 

Newly advocated variable selection method which fits on a linear programming context are the LASSO/AdaLASSO techniques, which consists in penalizing via the $\ell_1$-norm. Besides shrinking coefficients towards to zero, it has also the capability of effectively pushing coefficients to zero (an effect that ridge regression cannot achieve \cite{tibshirani1996regression}). The usage of LASSO and AdaLASSO in the QR context is the topic of study in \cite{li_l1-norm_2008,ciuperca_adaptive_2016,belloni_l1-penalized_2009,zou_regularized_2008,jiang_interquantile_2014}.
We refer to the reader the work from \cite{belloni_l1-penalized_2009}, where it is possible to find specific properties and convergence rates when using the LASSO to perform model selection in a QR framework. 

Regarding the penalization parameter $\lambda$, which dictates the shrinkage magnitude of the linear coefficients, the level of parsimony of the model can be defined by the user through such quantity. This is due to the fact that higher values of $\lambda$ means less variables selected to be non-zero. 

The single $\alpha$-quantile AdaLASSO is estimated by the following optimization problem:
\begin{IEEEeqnarray}{c}
\underset{\beta_{0},\beta}{\text{min}} \sum_{t \in T} \left( \rho_\alpha(y_t - \beta_0 -  \beta^T x_{t}) \right) +\lambda \sum_{p \in P} w_p | \beta_p |,\label{eq:l1-qar-optim} 
\end{IEEEeqnarray}
where $p$ is the collection of covariate indexes.
What differs the AdaLASSO from the LASSO is the inclusion of the term $w_p$. 
If the model (\ref{eq:l1-qar-optim}) is estimated with all $w_{p}=1$, the output of the optimization problem are coefficients of LASSO  $\beta^{L}_{p}$. The AdaLASSO coefficients $\beta^{AL}_{p}$ are obtained when solving the same optimization problem by letting $w_{p}=1/|\beta^{L}_{p}|$. 
% What differs the AdaLASSO from the LASSO is the inclusion of the term $w_p$. Suppose the model (\ref{eq:l1-qar-optim}) is estimated with all $w_{pj}=1$, the output of the optimization problem are coefficients of LASSO  $\beta^{L}_{pj}$. The AdaLASSO coefficients $\beta^{AL}_{pj}$ are obtained when solving the same optimization problem while letting $w_{pj}=1/|\beta^{L}_{pj}|$. % para voltar o 'j'







\section{conditional distribution based on Quantile Regression for Time Series}

In the previous section, we presented a linear model for estimating a single $\alpha$-quantile by using QR with adaLASSO as a regularization strategy to select the best model. However, to build a CDF from an array of quantiles, we propose to jointly estimate them, in order to explore the connection across different quantiles. 

Let the finite discretization of the interval $[0,1]$ be composed of a sequence of probabilities $0 < \alpha_1 < \alpha_2 < \dots < \alpha_{|J|} < 1$ and denote as $A$ the set $A = \{ \alpha_j \mid j \in J \}$, where $J$ is an index set for the probabilities $\alpha$. 
The $\alpha$-quantiles are, from this point forward, indexed by $j$, to account for the different models that are simultaneously estimated. A property that must be respected is the monotonicity of the quantile function $Q$, such that $q_{\alpha_1} \leq q_{\alpha_2} \dots \leq q_{\alpha_{|J|}}$.
The sequence of quantiles define a continuous quantile function after interpolation, and finally a CDF after inverting the estimated quantile function.




As we are interested on the conditional distribution as a whole, we estimate multiple quantiles at once. In order to produce a distribution function, the output of the problem must respect certain properties, such as being monotonically increasing. 
Besides that, one can expect that the value of similar quantiles be produced by similar models. If the coefficient of a given $p$ covariate changes too abruptly with respect to a change on the probability $\alpha$, there is a high probability that the estimation did not produce good results, capturing noise on the data.  In order to correct this much common behavior in QR estimation, we introduce a second derivative filter,given by the discrete approximation shown below:
\begin{equation}
D_{\alpha_j}^{2} \beta_{pj} := \frac{\left(\frac{\beta_{p,j+1}-\beta_{pj}}{\alpha_{j+1}-\alpha_{j}}\right)-\left(\frac{\beta_{p,j}-\beta_{p,j-1}}{\alpha_{j}-\alpha_{j-1}}\right)}{\alpha_{j+1}-\alpha_{j-1}}. 
\end{equation}
With this approach, one can keep track of the crossing quantiles issue \footnote{Definição de Crossing Quantile} % todo crossing quantile
as well as using a interquartile structure as a strategy to reduce noise on estimation % todo  explicar melhor o porquê 
The works by \cite{zou_regularized_2008, jiang_interquantile_2014} also use multiple quantile regressions at once and make use of interquantile similarities to produce regularization on the quantiles. In \cite{zou_regularized_2008}, the author uses the norm $\| \beta \|_{1\infty}=\sum_{p=1}^{|P|} \max\{ |\beta_j^{(k)} |\}$ as penalization. Such penalization is imposed on the maximum value among all quantiles for a given covariate. This idea is extended by \cite{jiang_interquantile_2014}, that uses a fused AdaLASSO mixing the LASSO penalization with the absolute interquantile difference.

The statistical model \textbf{Quantile Regularized Adaptive LASSO (QRAL)} is defined by the vector of coefficients $\beta_{0}$ and the matrix of size $|P| \times |J|$ of regressor coefficients $\beta_{pj}$. These coefficients are the solution of the minimization problem given below:
% \begin{IEEEeqnarray}{lr}  % para duas colunas
% 	\underset{\beta_{0j},\beta_j}{\text{min}} \sum_{j \in J} \left( \sum_{t\in T}\rho_{\alpha_j}(y_{t}-(\beta_{0j} + \beta_j^T x_t)) \right.  \span \nonumber \\
% 	\span \left. + \lambda\  \sum_{p \in P} w_{pj}^\delta \mid  \beta_{pj} \mid \right) + \gamma \sum_{p \in P} \sum_{j \in J'} |D^2_{pj}|,\label{eq:adalasso_model}
% \end{IEEEeqnarray}
\begin{IEEEeqnarray}{lr} % para uma coluna
  \underset{\beta_{0j},\beta_j}{\text{min}} \sum_{j \in J} \left( \sum_{t\in T}\rho_{\alpha_j}(y_{t}-(\beta_{0j} + \beta_j^T x_t)) \right. \span \\  
  \span \left. + \lambda\    \sum_{p \in P} w_{pj}^\delta \mid  \beta_{pj} \mid \right) + \gamma \sum_{p \in P} \sum_{j \in J'} |D^2_{\alpha_j}\beta_{pj}|, \label{eq:adalasso_model_mat1}\\
  \text{subject to} \span \nonumber \\
	\beta_{0j} + \beta_{j}^T x_{t} \leq \beta_{0,j+1} + \beta_{j+1}^T x_{t},& \forall t \in T, \forall j \in J_{(-1)}, \label{eq:adalasso_model_mat2} 
\end{IEEEeqnarray}
where the weights $w_{pj} = 1/\tilde{\beta}_{pj}$ and $\tilde \beta_{pj}$ are the coefficients from the first-step LASSO estimation. The parameter $\delta$ is an exponential parameter usually set to 1.
The sum of absolute values that compose the second derivative filter $\sum_{j \in J'}\sum_{p \in P}|D_{\alpha_j}^{2}\beta_{pj}|$ is added on the objective function multiplied by a tuning parameter $\gamma$, where the set $J'=\{2,\dots,|J|-1 \}$.


%
%% ===== Sec. III - Implementation ===== %
%
\section{Estimation, Evaluation and Simulation procedure} \label{sec:estimation-evaluation-simulation}

This section presents computational aspects of the estimation of our proposed model, such as the mathematical programming formulation of both the linear and the nonparametric models and presenting the evaluation metric. 
Finally we also show how to simulate future scenarios of wind power time series.
The methodology is implemented in R \cite{rlanguage2008} and Julia \cite{bezanson2012julia} languages (relying heavily on the packages JuMP \cite{DunningHuchetteLubin2017}, Gurobi, RCall and Dierckx) and using the Gurobi solver. 



\subsection{Estimation of the QRAL model} \label{sec:qral-estimation}

At first, all covariates must be normalized. 
If they are not in the same scale, the shrinkage feature of the lasso will fail, as different variables may have different weights according to their relative size.

Let $\tilde x_{t,p}$ be an input observation at time $t$ of covariate variable $p$.
The normalization process is a linear transformation to each covariate $p$ such that all have mean $\mu = 0$ and variance $\sigma^2 = 1$. 
We apply the transformation ${x}_{t,p} = (\tilde x_{t,p} - \bar{x}_{p}) / \hat\sigma_{\tilde x_{p}}$, where $\bar{x}_{p}$ and $\hat{\sigma}_{\tilde x_{p}}$ are respectively covariate $p$'s unconditional mean and standard deviation. The response variable $Y$ does not need to be transformed.

The QRAL model, as described in problem (\ref{eq:adalasso_model_mat1})-(\ref{eq:adalasso_model_mat2}), can be implemented as a linear programming problem as shown below:
\begin{IEEEeqnarray}{lr}
	\underset{\beta_{0},\beta,\varepsilon_{t j}^{+},\varepsilon_{t j}^{-}}{\text{min}} \sum_{j \in J} \sum_{t \in T}(\alpha_j\varepsilon_{t j}^{+}+(1-\alpha_j)\varepsilon_{t j}^{-}) \span \nonumber  \\
	\span + \lambda \sum_{p \in P} \sum_{j \in J} w_{pj} (\xi^+_{pj} + \xi^-_{pj}) \nonumber \\ 
	\span + \gamma \sum_{p \in P} \sum_{j \in J'} (D2_{pj}^+ + D2_{pj}^-),  \label{eq:adalasso-1} \\
	\mbox{subject to:} \nonumber & \\
	\varepsilon_{t j}^{+}-\varepsilon_{t j}^{-}=y_{t}-\beta_{0 j}-\beta_{j}^T x_{t},& \forall t \in T ,\forall j \in J,\\
	\xi_{pj}^+ - \xi_{pj}^- = \beta_{pj},&\forall p \in P, \forall j \in J\\ 
	D2_{pj}^+ - D2_{pj}^- = \frac{\left(\frac{\beta_{p,j+1}-\beta_{pj}}{\alpha_{j+1}-\alpha_{j}}\right)-\left(\frac{\beta_{p,j}-\beta_{p,j-1}}{\alpha_{j}-\alpha_{j-1}}\right)}{\alpha_{j+1}-\alpha_{j-1}}, \span   \nonumber \\
	\span \forall p\in P, \forall j \in J',  \\
	\beta_{0j} + \beta_{j}^T x_{t} \leq \beta_{0,j+1} + \beta_{j+1}^T x_{t},&\forall t \in T, \forall j \in J_{(-1)}, \label{eq:qral-crossing} \\
	\varepsilon_{t j}^{+},\varepsilon_{t j}^{-}\geq0,&\forall t \in T, \forall j \in J,\\
	\xi_{pj}^+, \xi_{pj}^- \geq 0, & \forall p\in P, \forall j \in J, \\
	D2_{pj}^+, D2_{pj}^- \geq 0, & \forall p\in P, \forall j \in J', \label{eq:adalasso-ult} 
\end{IEEEeqnarray}
where $J_{(-1)} = \{ 2, \dots, |J| \}$ is the set which contains all indexes but the first and $J'  = \{ 2, \dots, |J|-1 \}$ is the set which contains all indexes but the first and the last.
Variables $\varepsilon^+_t$ and $\varepsilon^-_t$ represent the quantities $|y-q(\cdot)|^+$ and $|y-q(\cdot)|^-$, respectively. The first line on the objective function represents the sum of the check function over all $j$: $ \rho_{\alpha_j}(y-q(\cdot)) = \alpha_j \varepsilon^+_{tj} + (1-\alpha_j) \varepsilon^-_{tj}$. The constraint (\ref{eq:qral-crossing}) assures that the quantile function be monotonic by forcing that, for every $x_t$ and $\alpha_j$-quantile, $q_{\alpha_{j}}(x_t) \leq q_{\alpha_{j+1}}(x_t)$.
The second derivative term $D^2_{\alpha_j}\beta_j$ is implemented on the optimization problem by adding a penalty on the objective function to penalize its absolute value, modeled as the sum of auxiliary variables $D2_{pj}^+ + D2_{pj}^-$. The tuning parameter $\gamma$ controls how rough the sequence $\{\beta_{pj}\}_{j \in J}$ can be, for a given $p$.

For a thin grid of value of parameters $\lambda$ and $\gamma$ given as input, we estimate a different model $m$ with coefficients $\beta_{0j}^m$ and $\beta_j^{Tm}$. For each $m$, its performance is evaluated according to two metrics presented on sections \ref{sec:SIC} and \ref{sec:GFS}.
The optimal parameters $\lambda^*$ and $\gamma^*$ for a given criteria are the ones that minimizes the evaluation metric.


% \subsection{Time-series Cross Validation} \label{sec:cv}

% Estimating the QRAL involves parameters $\lambda$ and $\gamma$, which should be known \textit{a priori}. In statistics and machine learning, a popular technique is using cross-validation (CV) to select the best value of parameters from the range of possibilities. How to select their values among this range is a crucial point in our methodology, as the estimated coefficients vary considerably with respect to parameter choice.

% Out of the different possible implementations of CV, we use the $\mathcal{K}$-fold CV. It consists in first partitioning the dataset in $\mathcal{K}$ equally sized sets, which are the $\mathcal{K}$ folds. For each fold $k \in \{1,\dots,\mathcal{K}\}$, the remaining $\mathcal{K}-1$ folds are used to estimate the model using parameter $\theta$ (for the QRAL model, $\theta = [\gamma \quad \lambda]^T$) and predicting the values in fold $k$. The error function $MAPE_\theta$ measures the result of this prediction.
% So, the CV error is given by the sum of all folds, for a given model which uses the vector of parameter $\theta$ is given by
% \[
%  CV(\theta) = \sum_{k \in \mathcal{K}} \sum_{j \in J} MAPE_\theta.\label{eq:cv-error}
% \]
% The optimum parameter $\theta^*_{CV}$, according to this methodology, is the value of $\theta$ which minimizes the CV error
% \begin{equation}
% \theta_{CV}^* = \argmin_\theta CV(\theta) .\label{eq:cv-equation}
% \end{equation}

% The usage of CV is not straightforward when data is dependent, which is the case of time series. As it is time dependent, one can be interested in using either all observations or to take the dependence away to not interfere on the estimation. The works
% \cite{bergmeir_note_2017} and \cite{bergmeir_use_2012} deals specifically with the usage of CV in a time series context. They provide tests with both $\mathcal{K}$-fold CV and $\mathcal{K}$-fold with non-dependent data. Both schemes are shown of Figure \ref{fig:cross-validation-scheme}.
% \begin{figure}
% 	\centering
% 	\includegraphics[width=0.7\linewidth]{Images/Cross-validation-scheme}
% 	\caption{$\mathcal{K}$-fold CV and $\mathcal{K}$-fold with non-dependent data. Observations in blue are used to estimation and in orange for evaluation. Note that non-dependent data does not use all dataset in each fold. Image from \cite{bergmeir_note_2017}.}
% 	\label{fig:cross-validation-scheme}
% \end{figure}
% In both settings, the training data is randomly split into a collection of sets $S_k$, forming a $\mathcal{K}$ size partition. Each of these $S_k$ is used as test set, while the rest is used to estimate coefficients which will be used to predict values of $S_k$. 
% As there are $\mathcal{K}$ folds, this procedure is done $\mathcal{K}$ times. 
% So, for a given vector of tuning parameter $\theta$, the CV score is given by the sum of the error function for each fold. 
% As the CV score is nonconvex, the optimization in (\ref{eq:cv-equation}) is done by iterating over a sequence of values in a thin grid and choosing the smallest one.




% Even though CV is very popular and produce great results, selecting model with Information Criteria involves less computational time. For the case where the selected model is very similar, it might be the case that the estimation methodology may change a little bit. It is definitely a topic that worths researching.


% \todoi{Ver se novas figuras (R/grafico-cv.r) e ver se incluir outras formas de CV} % escolhemos trabalhar apenas com este tipo de CV





\subsection{Scenario generation} \label{sec:scenario-generation}

This section presents how to generate future scenarios of time series $y_t$ from the estimated coefficients from a QR model. 
%Let $|T|$ be the total length of $\{y_t\}$ and $S$ the number of scenarios of size $K$ we produce. 
%The variables chosen to compose $x_t$ can be either exogenous variables, autoregressive components of $y_t$ or both. We use a nonparametric approach which to estimate, at every $t$, the $k$-step ahead conditional density of $y_t$.
To produce $S$ different future scenarios $\{ \hat{y}_{\tau,s} \}_{\tau=|T|+1}^{|T|+K}$, we use the following procedure:

\noindent\rule{\columnwidth}{3pt}

Procedure for simulating $S$ future scenarios of $\{y_{\tau,s}\}$

\noindent\rule{\columnwidth}{1pt}

\begin{enumerate}
	
	\item Estimate a QR model (for the QRAL solve the optimization problem defined in equation (\ref{eq:adalasso-1})-(\ref{eq:adalasso-ult})). 
	A sequence of coefficients $\{ \hat\beta_{0j} \}_{j \in J}$ and $\{ \hat\beta_{j} \}_{j \in J}$ is the output from this problem. 

	\item Initialize time index $\tau = |T| + 1$.
	
	\item For each scenario $s \in S$, do:
		\begin{enumerate}

		\item Let $x_{\tau,s} = [y_{\tau-1,s}, \dots, y_{\tau-12,s}]$ be the vector of explanatory variables used as input to predict the conditional distribution function in time $\tau$ and scenario $s$.

		\item Let $\tilde{Q}_{y_{\tau,s}|X}:A \times \mathbb{R}^d \rightarrow \mathbb{R}$ be the discrete quantile function. Its values are mapped according to the estimated quantile $\tilde Q_{y_{\tau,s}|X}(\alpha_j, x_{\tau,s}) \leftarrow \hat\beta_{0j} + \hat\beta_j^T x_{\tau,s}$, for all $j \in J$.
		
		\item In order to define the continuous function $\hat{Q}_{y_{\tau,s}|X}:[0,1] \times \mathbb{R}^d \rightarrow \mathbb{R}$ from $\tilde Q_{y_{\tau,s}|X}$, use linear interpolation connecting the points. As $0 < \alpha_1 < \cdots < \alpha_{|J|} < 1$, there are no quantile estimates for the intervals $[0,\alpha_1]$ and $[\alpha_{|J|},1]$. These gaps are filled by linearly extending the line that connects $\alpha_1$ to $\alpha_2$ on the left hand side and extending the line that connects $\alpha_{|J|-1}$ to $\alpha_{|J|}$ on the right hand side until the support $[0,1]$ is fully mapped.  

		% \item In any given period $\tau$, for every $\alpha \in A$, we estimate $q_{\alpha_{j}}$, for every $j \in J$.
		% Note that $x_{\tau}$ is supposed to be known at time $\tau$\footnote{In the presence of exogenous variables that are unknown, it is advisable to incorporate its uncertainty by considering different scenarios. In each scenario, though, $x_{\tau}$ must be considered fully known.}.
		
		% \item Let $\hat{Q}_{y_{\tau,s}|X}(\alpha,x_\tau)$ be the estimated quantile function of ${y}_{\tau,s}$. 
		% At first, we define a discrete quantile function $\tilde{Q}_{y_{\tau,s}}$. By mapping every $\alpha \in A$ with its estimated quantile $\hat{q}_{\alpha_j}(x_t)$, we define function $\tilde{Q}_{y_{\tau,s}}$. In order to produce a continuous function from a set of ordered points, we use linear interpolation and we arrive on the Quantile function $\hat{Q}_{y_{\tau}}$.
		
		%This process is described in more details on section \ref{sec:estimating-distribution}. 
		\item Let $U$ be a random variable with uniform distribution over the interval $[0,1]$. By using the result of the probability integral transform (PIT), random variable $F^{-1}_{y_{\tau,s}}(U)$ has the same distribution as $y_{\tau,s}$. The value of $y_{\tau,s}$ is built by drawing one random observation of U and applying the PIT. \todo{item cristiano}



		 \end{enumerate}
	% let $x_{\tau,s} = [y_{\tau-1,s}, \dots, y_{\tau-12,s}]$ be the vector of explanatory variables, used as input to predict the conditional distribution function in time $\tau$ and scenario $s$.
	
	
	\item Let $\tau = \tau + 1$. If $\tau > K$, then stop. Else, go back to step 3) . 


\end{enumerate}

\noindent\rule{\columnwidth}{1pt}


\subsection{Schwarz Information Criteria for Quantile Regression (SIC)} \label{sec:SIC}

Information criteria (IC) is the state of the art in time series model selection. It is also employed in other multiple quantile model studies \cite{zou_regularized_2008, jiang_interquantile_2014} to tune parameters.
An IC summarizes two aspects: the first refers to how well the model fits the in-sample observations, while the other part penalizes the quantity of covariates used as input.
By penalizing the model's size, we prevent overfitting from happening. Hence, in order for a covariate to be included in the model, it must supply enough goodness of fit.
The equation of SIC, applied for quantile autoregression case, is presented below:
% {\small
% \begin{align} 
% \begin{split}
% SIC_m = \sum_{j \in J} \left( \log \left(\sum_{t \in T}\rho_{\alpha_j}(y_t - \beta_{0j} - \beta_j^T x_t) \right) +  \frac{\log(|T|)|\epsilon_j|}{2|T|}  \right),\label{eq:SIC}
% \end{split}					
% \end{align}} 
 \begin{equation} 
\small
SIC^m = \sum_{j \in J} \left( \log \left(\sum_{t \in T}\rho_{\alpha_j}(y_t - \beta_{0j}^m - \beta_j^{Tm} x_t) \right) +  \frac{\log(|T|)|\epsilon_m|}{2|T|}  \right),\label{eq:SIC}
\end{equation}
where $\epsilon_m$ is the elbow set, defined as $\epsilon_m = \{(t): y_t - q(x_t) = 0 \}$. The authors in \cite{li_l1-norm_2008} show that the quantity $|\epsilon_m|$ is the effective degrees of freedom in the quantile regression.




\subsection{Model selection based on goodness of fit for scenarios (GFS)} \label{sec:GFS}

Having scenario simulation as a goal in our work demands models which perform well in this context, in opposition of the common approach of assessing performance for the $k$-step forecast.  Most statistical models are designed to provide a good fit on point forecasts, but produces scenarios which quickly converge to the unconditional mean.
The applications that use future scenarios, discussed in Section \ref{sec:introduction}, demands that scenarios resemble the time series past behavior.

We propose a novel metric, where we simulate future scenarios and take the mean absolute error (MAE) of the unconditional forecasted scenarios to select the model that produces scenarios the most similar to reality.
The MAE is defined by
\begin{equation}
MAE_{\theta}= \frac{1}{|J|} \frac{1}{12} \sum_{j \in J} \sum_{i = 1}^{12}  \left| q_i^{\alpha_{j}}- \hat q_i^{\alpha_{j}}  \right|,
\label{eq:MAE}
\end{equation}
where $\theta = [\gamma \quad \lambda]^T$, $q_t^{\alpha_{j}}$ is the true $\alpha_j$-quantile  and $\hat q_t^{\alpha_j}$ the predicted $\alpha_j$-quantile.
The MAE error function emphasizes the correctness across quantiles. Depending on the application, it might be interesting to put different weights on different quantiles. In this work, however, we will treat every quantile as equals concerning the error measure.

We consider the third year ahead as the unconditional state of the time series. Given that the observed values lies on $t \in \{1,\dots,|T| \}$, the subset of future scenarios $\{y_{|T|+25,s}, y_{|T|+26,s}, \dots, y_{|T|+36,s} \}$ is compared with the observed past values of each month. 
For each month $i = 1,\dots, 12$, take the $\alpha_j$-quantile $\hat q_t^{\alpha_j}$ from the set of $S$ scenarios $y_{|T|+24+i}$. The estimated quantile is compared with the historic quantile from this period. For example, we evaluate how close the 25\%-quantile
of all simulated scenarios in the month of March (from the third year) is against the 25\%-quantile
of all observed months of March. We take the MAE of the absolute difference $\left| q_i^{\alpha_{j}}- \hat q_i^{\alpha_{j}}  \right|$ for every month $i$ and quantile $j$ as the error measure, as shown on equation (\ref{eq:MAE}).

The output of problem (\ref{eq:adalasso-1})-(\ref{eq:adalasso-ult}) is the best 1-step ahead model, giving the tuning parameter $\lambda$ and $\gamma$. 
So, selecting tuning parameters according to realistic scenario generation puts together a goodness of fit for both the conditional $k$-step ahead, which produces reliable scenario on the short term, but also for scenario generated for the long term.  

% where $q_t^{\alpha_{j}}$ is the true $\alpha$-quantile from the data (in the case study, we use the monthly distribution as a good enough approximation of the true quantile, as RG time series such as wind power are stationary) and $\hat q_t^{\alpha_j}$ is the $\alpha$-quantile from these scenarios, when estimating the model with parameters $\lambda$ and $\gamma$.
% This function has the advantage of penalizing error proportionally to the quantile value it is estimating. 


% In order to evaluate the model performance, we need to define an error metric. The minimization of this error metric is the objective in estimating the statistical model. 
% As conditional distribution is the focus in this paper, 

% \todo{Voltar para essa seção}





%


%% ===== Sec. VI - Case studies ===== %
%
\section{Case Studies}


\subsection{Controlled Studies I - Autoregressive Process} \label{sec:ar-study}


In our simulation exercise the performance of QRAL model is evaluated by simulating data from an AR(1) model and then trying to recover its quantiles using QRAL and other competing models, namely:
\begin{enumerate}
\item \textbf{Quantile Regularized Adaptive LASSO (QRAL)}, which estimates a different model for each quantile $Q_{y_t|X}(\alpha,\cdot)$, for all ${j \in J}$. In practice, it means that each coefficient $\beta_{1j}$ is estimated with regularization on each quantile. %As the QRAL estimates a different solution for all every $\alpha$-quantile of $y_t$, the model $$y_t^{\alpha_j} = \beta_{0\alpha_j} + \beta_{\alpha_j} y_{t-1}, \quad \text{for all } j \in J$$ will produce as output a different model for each probability quantile $y_t^{\alpha_j}$.
\item \textbf{Quantile Regression as Koenker (QRK)} as originally proposed by \cite{koenker1978regression}, where each coefficient $\beta_{1j}$ is estimated independently using QR. 
\item A simple \textbf{Autoregressive (AR)} process.
\end{enumerate}

The AR(1) specification is given by:
\begin{equation}
y_t = \beta_0 + \beta_1 y_{t-1} + \varepsilon_t, \quad \varepsilon_t \sim N(0, 1), \quad t=1,\dots,400 \label{eq:ar1}
\end{equation}
with $\beta_0 = 0$, $\beta_1 = 0.3$ and $y_0 = 0$. This length is chosen due to the fact that the sample size of RG time series have such size in general.

The inter-quantile regularization parameter $\gamma$ (see Eq.(\ref{eq:adaLASSO-1})-(\ref{eq:adaLASSO-ult})) is estimated using cross-validation, a popular technique to select the best value of parameters for cross-sectional data. 
Since in this experiment the model has only one lag, model selection will not be evaluated, hence $\lambda=0$.
After simulating 1000 different time series given by Equation (\ref{eq:ar1}), the three models are estimated.
% TODO referência cross-validation

Since the main objective of this simulation experiment is to evaluate how our nonparametric model can correctly recover the true AR(1) process, model performance can be evaluated by examining how closely the estimated quantiles are from the populational ones. The results for each model are depicted in Figure \ref{fig:boxplot-ar1}, where a Boxplot containing the results for the 1000 simulations is shown. %a single boxplot for AR(1) and one for each probability $\alpha$ for QRAL and QRK.
\begin{figure}[ht]
	\centering
	\includegraphics[width=1.0\linewidth]{Images/boxplot-ar1.pdf}
	\caption{Boxplot showing estimated coefficient after 1000 iterations. On the left hand side, the boxplot of the AR(1) coefficient estimation. Note that for the AR(1) the coefficient is equal for all probabilities $\alpha$. On the right hand side, the boxplot of the regular QR (where $\gamma = 0$) and the QRAL where $\gamma$ is selected using cross-validation. }
	\label{fig:boxplot-ar1}
\end{figure}
The conclusions from this experiment are: (i) coefficient estimation errors for the central quantiles are not far from those estimated by the AR model; (ii) extreme quantiles are usually harder to estimate, due to having fewer observations, and as a consequence, the estimation error increases on the extremes (iii) QRAL has an advantage over QRK by showing smaller variance of estimators.


\subsection{Case study with real data}

In this section, the QRAL methodology is tested in generating future scenarios for a real RG time series. The wind power time series, measured in megawatts, is composed of 31 years (from 1981 to 2011) of monthly observations from a wind farm located in the Brazilian Northeast. % Figure \ref{fig:icaraizinho-mensal} depicts a time series plot from which a strong seasonal pattern can be seen. 
The annual seasonality is seen in Figure \ref{fig:icaraizinho-mensal}, where each individual year is plotted as a single line on the graph. 
\begin{figure}[ht]
\centering
\includegraphics[width=0.8\linewidth]{Images/icaraizinho-mensal2.pdf}
\caption{Icaraizinho annual data. Each series consists of monthly observations for each year.}
\label{fig:icaraizinho-mensal}
\end{figure}

We use four different models to generate scenarios for monthly wind power time series:  QRAL, QRK, QRL (Quantile Regularized LASSO is essentially the same model as QRAL in which we let $w_{ij} = 1$, for all $i$ and $j$) and SARIMA. 
The tuning parameters $\lambda$ and $\gamma$ of QRL and QRAL are selected according to the two metrics presented on Section \ref{sec:estimation-evaluation-simulation}: SIC and GFS.

The estimated coefficients for the QR based models are presented in Figures \ref{fig:betas-qrk}-\ref{fig:betas-sic}. 
\begin{figure}[ht]
	\centering
	\includegraphics[width=1.0\linewidth]{Images/Betas-icaraizinho-QRK.pdf}
	\caption{Estimated coefficients for the QRK model.}
	\label{fig:betas-qrk}
\end{figure}
\begin{figure}[ht]
	\centering
	\includegraphics[width=1.0\linewidth]{Images/Betas-icaraizinho-GFS.pdf}
	\caption{Estimated coefficients for the QRAL (GFS) model.}
	\label{fig:betas-gfs}
\end{figure}
\begin{figure}[ht]
	\centering
	\includegraphics[width=1.0\linewidth]{Images/Betas-icaraizinho-SIC.pdf}
	\caption{Estimated coefficients for the QRAL (SIC) model.}
	\label{fig:betas-sic}
\end{figure}
We show for QRAL and QRK coefficients estimates for obtained via selection criteria. For each of these models, $\beta_0(\alpha)$ is shown on the left hand side figure, while $\beta(\alpha)$, for each lag, is on the right side.
The biggest difference between QRK and QRAL (via SIC) is that the former presents $\beta$ coefficients which are noisier and have higher variability between quantiles (as seen on experiment of section \ref{sec:ar-study}), given that, for this model, for each quantile $j$ a completely separated estimation problem is solved. Our proposed model has the advantage that quantiles are jointly estimated in a single model, which helps to decrease variance of estimators. Regularization of covariates plays a larger role in QRAL (via GFS criteria), where fewer covariates are left nonzero.

We also estimate a SARIMA model using the \emph{forecast} \cite{hyndman2008forecastpackage} package from R software, via the \emph{auto.arima} function, which selects the best model according to an IC (details on \cite{hyndman2008forecastmanual}). Even though the fully automatic estimation procedure did not select seasonal differencing, we force this operation in order to produce better scenarios according to the GFS metric. The best SARIMA model is $(2,0,2)\times(2,1,0)$ with drift and monthly seasonals.


The resulting tuning parameters estimates followed by SIC and GFS metric values for all models are shown on Table \ref{tab:results-icaraizinho}. When comparing the performance  of QRAL and QRL, we see that for either metric the adaptive LASSO produces a better estimation. 
QRAL, regularized via SIC, performs better than QRK for both metrics, but produces worse scenarios than SARIMA. There is a trade-off between having a good model for the short term (SIC metrics) and having a good model for producing scenarios on the long term (GFS metrics). 
When choosing the GFS metric, that has a better fit for generating scenarios, the chosen model must be able to select important features over the long term. On the other hand, when the model is tuned via SIC, more variables are selected and as a result the model is able to capture better short term fluctuations.
With selection by GFS, the size of the regularization parameters are bigger than those obtained via SIC, and the effect of larger parameters is seen clearly when comparing Figure \ref{fig:betas-gfs} with Figure \ref{fig:betas-sic}, as the coefficients shrinkage is much stronger on QRAL (GFS) than in QRAL (SIC). It is interesting to notice that the best QRAL model according to the metric GFS has fewer nonzero coefficients than that selected by optimizing SIC, but has the best results in minimizing the MAE with respect to the historic quantiles.  
\begin{table}[ht]
	\centering
	\caption{Cumulated statistics across all $\alpha_j$ quantiles}
	\label{tab:results-icaraizinho}
	\begin{tabular}{|l|c|c|c|c|}
		\hline 
		Method (tuning criteria) & $\lambda$ & $\gamma$ & SIC & GFS\tabularnewline
		\hline 
		\hline 
		QRL   (SIC) & 0.75 & 0.8 & 12.18 & 5.43\tabularnewline
		\hline 
		QRAL   (SIC) & 0.25 & 0.8 & 12.15 & 5.41\tabularnewline
		\hline 
		QRL    (GFS) & 9.02 & 33.1 & 13.68 & 4.58\tabularnewline
		\hline 
		QRAL   (GFS) & 1.82 & 90.01 & 13.63 & 4.23\tabularnewline
		\hline 
		QRK    ( - ) & 0 & 0 & 13.00 & 5.92 \tabularnewline
		\hline 
		SARIMA ( - ) & - & - & - & 5.15 \tabularnewline
		\hline 	
	\end{tabular}
	\end{table}
	

% The accuracy of the generated scenarios - considering the MAE metric - in recovering the historic quantiles for each month is the metric used for evaluation.

% In Figure \ref{fig:simulated-quantiles}, we compare three specific quantiles (5\%, 50\% and 95\%) of the historic data with SARIMA and QRAL (with tuning by GFS) over the 12 months of the third simulated year, as explained on the section  \ref{sec:GFS} about GFS. 
% \begin{figure}[h]
% 	\centering
% 	\includegraphics[width=1.0\linewidth]{Images/Comparison-scenarios-icaraizinho.pdf}
% 	\caption{Estimated quantiles 5\%, 50\% and 95\% of SARIMA and QRAL in comparison with historic quantiles.}
% 	\label{fig:simulated-quantiles}
% \end{figure}
In Figures \ref{fig:scenarios-qral} and \ref{fig:scenarios-sarima}, we compare the median (50\%), extreme quantiles (5\% and 95\%) and the 1st and 3rd quartiles (25\% and 75\%) obtained from  both QRAL and SARIMA models via simulation, with the associated historical quantiles. The period of simulation is the same employed for the metric GFS, that is for the year 2014. As it can be seen, the results from QRAL are better in capturing the unconditional seasonality than those produced by SARIMA, as the former's quantiles are closer to the historic quantiles than the latter's.


% \begin{figure}[h]
% 	\centering
% 	\includegraphics[width=1.0\linewidth]{Images/QRAL-historic-icaraizinho}
% 	\caption{Historic quantiles of icaraizinho}
% 	\label{fig:betas-gfs}
% \end{figure}
\begin{figure}[ht]
	\centering
	\includegraphics[width=1.0\linewidth]{Images/ScenariosHHH3-QRAL-icaraizinho.pdf}
	\caption{Comparison of unconditional generated scenarios (the third year of simulation) with historic quantiles of QRAL.  The areas represent QRAL quantiles ($5\%-95\%$),
	while the historic quantiles are marked by symbols, by month. }
	\label{fig:scenarios-qral}
\end{figure}

\begin{figure}[h]
	\centering
	\includegraphics[width=1.0\linewidth]{Images/ScenariosHHH3-SARIMA-icaraizinho.pdf}
	\caption{Comparison of unconditional generated scenarios (the third year of simulation) with historic quantiles of SARIMA. The areas represent SARIMA quantiles ($5\%-95\%$),
	while the historic quantiles are marked by symbols, by month. }
	\label{fig:scenarios-sarima}
\end{figure}



%% ===== Sec. - Conclusion ===== %
%
\section{Concluding Remarks}

In this work, we propose a nonparametric methodology based on QR to build the CDF of wind time series.  We develop a linear model with regularization both on the quantile and on the covariates to produce future scenarios of RG. These scenarios are input for various applications in power systems and are essential in measuring risk in energy trading, planning the expansion of the energy systems and the dispatch problem. The nonparametric framework for estimating the CDF is specially useful when the data is not distributed according to a known distribution. The scenarios generated outperformed other known benchmarks, both in the QR literature (the Koenker model) as in the classic time series framework (SARIMA). The results obtained in this work brings incentives continue researching methods for nongaussian time series, as they are present in many real world applications. 

% % 

\bibliographystyle{IEEEtran}
\bibliography{Thesis,QR,Bibhenriquinho}



\end{document}
